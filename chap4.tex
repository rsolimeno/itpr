% Chapter 4
\chapter{DIGIPEATERS AND NODES - THE BASICS}
\section{DIGIPEATERS}\index{digipeaters}
Digipeater is the term used to describe a packet radio digital repeater. Unlike the FM voice repeaters, most digipeaters operate on simplex and do not receive and transmit simultaneously. They receive the digital information, temporarily store it and then turn around and retransmit it. Your TNC can be used by others as a digipeater if you have the command \verb!DIGIPEAT!\index{DIGIPEAT} turned ON.

You use a digipeater by entering its callsign after a V or VIA in your connect sequence.\index{connect sequence} Here are some examples of proper connect sequences:
\begin{itemize}
\item \verb!C W6PW-3 V WB9LOZ-2!
\item \verb!C N6ZYX V WA6FSP-1,WD6EOB-3!
\item \verb!C W6ABY-4 V K6MYX,N2WLP-2,AB6XO!
\end{itemize}

In the first example, the sequence shown means: Connect to W6PW-3 via the WB9LOZ-2 digipeater.

Your TNC will allow you to enter up to eight digipeaters in your connect sequence or in your \verb!UNPROTO!\index{UNPROTO} path, but using more than 3 usually means long waits, lots of repeated packets and frequent disconnects, due to noise and other signals encountered on the frequency.

When entering the list of digipeaters in your connect sequence,\index{connect sequence} you must make sure that you enter them in the exact order that your signal will use them. You must separate the calls by commas, without any spaces, and the EXACT callsigns must be used, including the SSID, if any. That means you need to know what digipeaters are out there before you begin randomly trying to connect to someone. Turn \verb!MONITOR ON!\index{MONITOR} and watch for the paths that other stations are using.

Something to remember when using digipeaters is the difference between making a connection and sending information packets. If the path isn't all that good, you might be able to get a connect request\index{connect request} through, but will have a difficult time with packets after that. The connect request\index{connect request} is short so it has much less of a chance of being destroyed by noise or collisions than a packet containing information. Keeping information packets short (\verb!PACLEN!\index{PACLEN} set to 40 or less) can help keep retries down when the path is less than ideal.

\section{NODES}\index{NODES}
Net/Rom, TheNet, G8BPQ packet switch and KA-Node are names that refer to a device called a packet node. This is another means of connecting to other packet stations. Later on in this series you'll find a complete review of node operation, but for now we'll cover the basics so that you can begin to use the node network.\index{node network} The difference between a digipeater and a node that you should note here is that you connect to a node rather than using it in a connect path as you do with a digipeater. Some packet stations are set up so that they can be used as a digipeater and as a node.

First, you need to determine what nodes are located close to you. You can do this by monitoring and watching for an ID, or by watching to see what other stations in your area are using. It is most common for a node to have an alias ID in addition to its callsign. Once you determine the callsign or alias of a local node,\index{local node} you connect to it the same way as you connect to any other packet station. You may use either the callsign or the alias to make the connection. For example, the node I operate has the alias ID of BERKLY and the callsign of WB9LOZ-2, so you could connect to it using: 

\begin{flushleft}
\verb!C BERKLY! or \\
\verb!C WB9LOZ-2!
\end{flushleft}

\noindent Either one will work.

When you connect to a node, your TNC automatically switches to converse mode,\index{CONVERSE mode} just like when you connect to any packet station. Anything you now type is sent to the node as a packet, and the node acknowledges each packet back to your TNC. For the remainder of your connection your TNC works only with this one node.

To use the node network to connect to another local station, you simply connect to the node and then enter a connect request\index{connect request} to the other station. Say you wanted to connect to K9AT using the WB9LOZ-2 node. You first connect to WB9LOZ-2:

\begin{flushleft}
\verb!C WB9LOZ-2!
\end{flushleft}

\noindent and then, while you ARE STILL CONNECTED TO THE NODE, you enter the connect request\index{connect request} to K9AT:

\begin{flushleft}
\verb!C K9AT!
\end{flushleft}

The node will then retransmit your connect request\index{connect request} and you'll receive one of two responses: "Connected to K9AT" or "Failure with K9AT". Once you are connected you hold your QSO just as if you had connected direct or via a digipeater. When your QSO is finished, go to command mode on your TNC (Control C) and enter:

\begin{flushleft}
\verb!D <CR>!
\end{flushleft}

\noindent and you will be disconnected from the node and the station you were working.

Some nodes have a \verb!BYE! command available for disconnecting. You can get a list of available commands from any node by sending a question mark. All of the node commands\index{node commands} will be covered in detail later.

\begin{quotation}
NOTE: If the node you're using is a G8BPQ packet switch, it might have several frequency ports. You'll have to enter a port number between the C and the callsign in your connect request\index{connect request} to indicate the frequency you want to use, such as:
\end{quotation}

\begin{flushleft}
\verb!C 2 K9AT!
\end{flushleft}

\noindent Enter \verb!PORTS! for a list of the frequency ports available.

\section{NODE NETWORK}\index{node network}
The packet nodes work together to form a packet node network.\index{packet node network} Once an hour each node transmits a list of other nodes that it knows about. The neighboring nodes\index{neighboring node} use this information to keep track of the other nodes in the network. We will discuss how all of this works later on.

When you're connected to a node you can enter:

\begin{flushleft}
\verb!NODES <CR>! or \\ \index{NODES}
\verb!N <CR>!
\end{flushleft}

\noindent and you'll receive a list of other nodes that you can reach on the network from the node you're using. You'll note that the node list will vary in length and in the calls listed as you move from frequency to frequency, since all frequencies are not linked together. The list gives both an alias ID and a callsign for each node. The alias ID often gives you a hint as to where the node is located, but not always. To find out for sure where a node is located you'll need to get a copy of the descriptive node listings that are available on most packet bulletin board systems. These complete lists give the alias, callsign, location, frequency and other information on each node in the network.

To connect to a station in another area using the node network\index{node network} you first must determine which node is closest to the station you want to work. For demonstration purposes, let's say we want to connect to N6XYZ. He's told you he uses the W6ABC-3 node, so you check the node list and see that GOLD:W6ABC-3 is listed. WHILE YOU ARE STILL CONNECTED TO YOUR LOCAL NODE\index{local node} you connect to the distant node by sending a normal connect request,\index{connect request} in this case: 

\begin{flushleft}
\verb!C GOLD! or \\
\verb!C W6ABC-3!
\end{flushleft}

Your TNC will send this as a packet to your local node\index{local node} and your local node will acknowledge it. The network will then go to work for you and find the best path between your local node and the one you're trying to reach. Remember, with digipeaters you needed to know the exact sequence of stations. With nodes you don't. The network does that for you.

You might have to be a little patient here, since it sometimes takes a few minutes for the connection to be completed. Don't type anything while you're waiting for a response because any new information received by your local node\index{local node} will override any previously entered information. When the network has completed its work you'll receive one of two responses: "Connected to W6ABC-3" OR "Failure with W6ABC-3". If it can't connect for some reason, try again later. It could be that W6ABC-3 is temporarily off the air or the path has decayed and is no longer available. We're going to be positive here and say we received the first option and got connected to the node.

Once you're connected to W6ABC-3, enter:

\begin{flushleft}
\verb!C N6XYZ!
\end{flushleft}

Again, your TNC will send this as a packet to your local node\index{local node} and the local node will acknowledge it and send it down the path to W6ABC-3. W6ABC-3 will then attempt to connect to N6XYZ. Here again you'll get one of the two responses: "Connected to N6XYZ" OR "Failure with N6XYZ". If you get connected, you hold your QSO just as you normally would, but there's one BIG difference -- your TNC is receiving acknowledgements from your local node,\index{local node} and N6XYZ is receiving acknowledgements from W6ABC-3. The acknowledgements do not have to travel the entire distance between the two end stations. Each node in the path handles the acknowledgement with the next node in line. Because of this, retries are greatly reduced and your packets get through much faster than when using a similar number of digipeaters.

When you're finished with the QSO, you disconnect in the normal manner. Use the \verb!BYE! command, if available, or go to Command Mode on your TNC and enter:

\begin{flushleft}
\verb!D <CR>!
\end{flushleft}

Using either method, the entire path will disconnect for you automatically.

Nodes offer a variety of other features besides allowing you to connect to other stations. We'll look at those and go into much more detail on the packet network in chapters 10 and 11 of this series.