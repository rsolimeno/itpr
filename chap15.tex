% Chapter 15
\chapter{TIPS FOR BETTER PACKET OPERATION}
Here are some tips to help make your packet operating more enjoyable. Whether it's while making local QSOs, checking into a BBS or mailbox, or working DX, there are a few things you should take into consideration that will help eliminate problems and waiting time, increase your ``throughput" and make packet a lot more fun. (``Throughput" is a word that has come into common usage by packet operators and means the amount of usable packet information transmitted or received.)

When connecting to another station, don't use a digipeater or node unless you have to. Each digipeater you add to the path increases the time required to get your signal to its destination and to get an acknowledgement returned. It also increases the chance for interference and for collisions with other packets. You'll be amazed at the difference in throughput when comparing a direct connect to one with just one digipeater in the path.

The packet node network, as discussed in previous articles in this series, does a great deal to help you get your packets through, but you must remember that throughput there, too, is affected by the number of nodes used and by the conditions between you and the destination station. The big advantage of the nodes is that the acknowledgements do not have to return all the way from the destination station before your TNC is satisfied. Packets are acknowledged from node to node, so that eliminates a large part of the problems encountered. Getting the original packet through, however, remains to be as much of a problem for the nodes as it is for you when using digipeaters. It can take several minutes to get a packet through when you're working a station some distance away. Have patience!

Dr. Tom Clark, W3IWI, has determined that for EACH HOP in a packet path the loss of packets can vary anywhere from 5\% to 50\% depending on the amount of traffic. Remember, each digipeater and node adds a hop, compounding the problem, and you have twice as many hops as you might think, because of the acknowledgements. You can see how quickly the path deteriorates as traffic increases and digipeaters and nodes are added to it.

If you have a choice, use a frequency that doesn't have a lot of other traffic on it. It makes sense that the more stations there are on a frequency, the more chances there are for collisions and retries. A path that will work perfectly without a lot of traffic, can become totally useless under heavy traffic conditions. Just one additional station on the frequency can decrease throughput by about half in many cases.

Another consideration, especially if working over a long distance, is atmospheric conditions. You might not have experienced this before on VHF, but with packet's high sensitivity to noise, a slight change in signal strength can mean the difference between getting your packets through or not getting them through. Long paths between nodes are very susceptible to these changing conditions. There are times, especially on a hot summer day, when it's impossible to get a packet from one node to the other on what is normally a good path. At other times, ``thermals" can increase your range dramatically and you're able to use node paths that normally don't exist. In the San Francisco Bay Area, the fog has a significant affect on VHF signals. When a fog bank is moving in off the Pacific, it can act as an excellent reflector. Signals that normally aren't heard or are very weak can reach signal strengths of 40 over S9.

Multipath is another problem that can greatly affect your packet signal. Multipath is the term used to describe the receipt of multiple signals from one source due to reflections off of buildings, hills or mountains. The ``ghost" in a television picture is a form of multipath. A station with a very strong signal into a digipeater or node often cannot use that path if multipath causes the signal to be distorted. Each packet is checked for 100\% accuracy and is not acknowledged unless it is. Multipath reflection can cause occasional bits to be lost so you can end up with multiple retries and a poor path even with strong signals.

To sum up, for best results on VHF use the least number of digipeaters and nodes as possible, use a frequency with low activity, and be aware of atmospheric conditions and multipath problems. Remember, by decreasing \verb!PACLEN! and \verb!MAXFRAME! in your TNC, you improve your chances of getting packets through under poor conditions.

If you use packet on HF, remember to change your transmit baud rate to 300 and to use a short \verb!PACLEN! (a value of 40 seems to work quite well) and a \verb!MAXFRAME! of 1. The chances of getting a short packet through the noise and QRM are much better than for a long one.