% Chapter 18
\chapter{WB9LOZ'S PARTING COMMENTS}
In the previous 17 chapters, I have attempted to cover all of the basics of packet radio -- from setting up your TNC and making your first QSO, to using digipeaters, the packet node network, and bulletin board systems. Many of the TNC commands have been explained, including the best settings for normal use, and I've offered some suggestions that should make it easier and more enjoyable for you to use packet radio.

Now that you have the basics, you might want to continue with your study by investigating some of the other facets of packet radio. There are several programs available that I haven't covered in this series that you might find interesting. There's the Packet Cluster software used by the DX Spotting Network for finding those rare DX stations, APRS-the Automatic Packet Reporting System\index{APRS} that's now very popular for station locating and for use with GPS, the Global Positioning Satellites, J-NOS and T-NOS for use with TCP/IP, other networking programs like Tex-Net and Rose, and new computer programs specifically written for packet and the other digital modes. The list goes on and on. PAC-SAT,\index{PAC-SAT} the amateur packet satellite program, is growing in popularity as more satellites carrying packet radio equipment are released. High speed modems running at speeds of up to 56 kilobaud are just around the corner for general use on packet radio. What developments will be next?

To keep up with the latest developments in Packet Radio, join your local packet radio group or digital communications club. Become a member of TAPR,\index{TAPR} the Tucson Amateur Packet Radio Corporation, the national organization that is devoted to packet radio development and education. TAPR has a quarterly newsletter and offers kits, publications, and a disk library of software and information (including this ``Introduction to Packet Radio'' that you're reading). You can contact them at (817) 383-0000. Read the packet columns in ``QST'', ``CQ'', ``73'' and other ham magazines and look for bulletins on your local BBS offering new information and discussions of developing systems, software and hardware. Packet Radio, and digital communications in general, are still relatively new areas and I'm sure you'll be seeing lots of changes in the years ahead.

I'd like to thank the following people for their help in preparing this series: Don Simon, NI6A; Bill Choisser, K9AT; Don Fay, K4CEF; Scott Cronk, N7FSP; Roy Engehausen, AA4RE; and Hank Oredson, W0RLI. Their help and their answers to my questions are greatly appreciated.

If you have any comments on this ``Introduction to Packet Radio'' series or you would like to correct or update any of the information contained in this work, please please contact me through my web site, given below. I would enjoy hearing from you, and your comments would be very much appreciated. I hope that you've found this to be informative and helpful in making packet radio more enjoyable for you.\\

73, Larry Kenney, WB9LOZ\\
http://www.choisser.com/hamradio/packet.html 