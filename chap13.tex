% Chapter 13
\chapter{TNC COMMANDS - PART 2 OF 3}
In this chapter and the next we'll take a look at some of the TNC commands\index{TNC commands} available to you that we haven't covered previously. You might find that some of the commands are not available in your particular TNC or that they're used in a slightly different manner than what is presented here. Not all TNCs operate exactly the same. Please refer to your owner's manual for specific details on how to use these commands if they don't function as described here.

\section{8BITCONV}\index{8BITCONV} This command enables the transmission of 8-bit data in converse mode. Used with \verb!AWLEN! - see below. For normal packet operation, such as keyboard to keyboard transmissions, use of bulletin board systems, and the transmission of ASCII files, \verb!8BITCONV! should be OFF. If you need to transmit 8-bit data, set \verb!8BITCONV ON! and set \verb!AWLEN! to 8. Make sure that the TNC at the receiving end is also set up this way. This procedure is normally used for transmission of executable files or a special non-ASCII data set.

\section{AWLEN}\index{AWLEN} This parameter defines the word length used by the serial input/output port of your TNC. For normal packet operation, as described above, \verb!AWLEN! should be set to 7. Set to 8 only if you're going to send 8-bit data.

\section{AX25L2V2}\index{AX25L2V2} This command determines which level of AX.25 protocol you're going to use. If OFF, the TNC will use AX.25 Level 2, Version 1.0. If ON, the TNC will use AX.25 Level 2, Version 2.0. Note: Some early TNCs will not digipeat Version 2.0 packets. With \verb!AX25L2V2 OFF!, if your TNC sends a packet and the packet doesn't get acknowledged the first time it was sent, it will send it again and again, until an ``ack'' is received or the TNC retries out. With \verb!AX25L2V2 ON!, if your TNC sends a packet and doesn't receive an ``ack'' the first time, it will send a poll frame to see if the other TNC received the packet. If yes, then it would continue, if not then it would send the last packet again. The advantage here is that short poll frames are sent, rather than long packets containing data. This can greatly reduce channel congestion. For VHF/UHF operation, it is almost essential that every TNC have \verb!AX25L2V2 ON!. Many operators have suggested that Version 2.0 NOT be used on the HF bands as it tends to clutter the frequency with poll frames. See the \verb!CHECK! command below for related information.

\section{BEACON}\index{BEACON} Used with \verb!EVERY! or \verb!AFTER! to enable beacon transmissions.

\verb!BEACON EVERY n! - send a beacon at regular intervals specified by n.
\verb!BEACON AFTER n! - send a beacon once after a time interval specified by n having no packet activity on the frequency.\\

n = 0 to 250 - specifies beacon timing in ten second intervals: 1 = 10 seconds, 2 = 20 seconds, 30 = 300 seconds or 5 minutes, 180 = 1800 seconds or 30 minutes, etc.\\

Examples:\\

\verb!BEACON EVERY 180 (or B E 180)!: the TNC will transmit a beacon every 30 minutes.\\
\verb!BEACON AFTER 180 (or B A 180)!: the TNC will transmit a beacon after it hears no activity on the frequency for 30 minutes.\\
\verb!B E 0!: will turn the beacon off.\\

The text of the beacon is specified by \verb!BTEXT! and can contain up to 120 characters. The path used for the beacon transmission is specified by the \verb!UNPROTO!\index{UNPROTO} command. YOU SHOULD USE BEACONS INTELLIGENTLY! Beacons are often a point of controversy in the packet community because they tend to clutter the frequency if used too frequently. You should keep your beacons short and infrequent, and they should only be used for meaningful data. Bulletin boards use the beacon for advising the community of who has mail waiting for them, clubs use beacons for meeting announcements, and beacons are used for severe weather warnings. In areas with heavy packet activity, beacons should not be used just to let everyone know that you're monitoring the frequency, that your mailbox is ready, or that you'd like someone to connect to you. You should monitor the frequency for activity and make some connections yourself.

\section{CHECK n}\index{CHECK n} Sets a timeout value for a packet connection. When a connection between your station and another seems to ``disappear'' due to changing propagation, channel congestion or loss of the path, your TNC could remain in the connected state indefinitely. If the \verb!CHECK! command is set to a value other than 0, the TNC will attempt to recover the connection or disconnect. The action taken depends on the setting of \verb!AX25L2V2!. The value of \verb!CHECK (n)! may be set from 0 to 250 and the timing is based on the formula of n * 10 seconds. (n = 1 is 10 seconds, n = 5 is 50 seconds, n = 30 is 300 seconds or 5 minutes, etc. A value of 30 is a recommended value to use.) If \verb!CHECK! is set to 0, it disables the command. If \verb!AX25L2V2!\index{AX25L2V2} is ON, the TNC will send a ``check packet'' to verify the presence of the other station if no packets have been heard after (n * 10) seconds. If a response to the ``check packet'' is received, the connection will remain. If no response is received, the TNC will begin the disconnect sequence, just as if the DISCONNECT command had been sent. If \verb!AX25L2V2! is OFF, after no packets are heard for n * 10 seconds, the TNC will not send a check packet, but will begin the disconnect sequence.

\section{CMSG}\index{CMSG} Enables the automatic sending of a connect message whenever a station connects to your TNC. If \verb!CMSG! is ON, the TNC will send the message contained in CTEXT as the first packet of the connection. CTEXT can contain up to 120 characters. Of course, you must have a message in CTEXT for CMSG to function. This feature is often used when the station is on but the operator is not present. The connect message is used to advise the other station of that fact, and often says to leave a message in the TNC buffer or mailbox. If \verb!CMSG! is OFF, the CTEXT message is not transmitted.

\section{KISS}\index{KISS} Enables the TNC to act as a modem for a host computer, allowing programs such as TCP/IP, the G8BPQ Packet Switch, various BBS programs, and other programs using the Serial Link Interface Protocol (SLIP) to be run. Before turning \verb!KISS! on, set the radio baud rate and terminal baud rate to the desired values. Set \verb!KISS! to ON and then issue a \verb!RESTART!\index{RESTART} command.
